\section{Database Systems for Cooperative Design}
\label{sec:cooperativedb}
In this section, we summarize how cooperative design
process on CAD systems can be realized, and consider the requirements for 
database systems to support cooperative design.
We will also mention functions of conventional CAD database systems 
for cooperative design and their problems.

\subsection{Cooperative Design Processes}
\label{subsec:coop-process}
Cooperative design process consists of two phases, first top-down design
and then bottom-up design.
Let us consider a case to design a lare system.
%Cooperative design process usually goes in a bottom-up manner, 
%following in a top-down manner.
%Here, for an example, we consider the case of designing a system.
%Figure ?? shows a typical process of design of the system.

At first, the system to be designed is recursively divided into
some subsystems.
In most cases, each subsystem is a functional block which can be 
designed independently from each other.
In cooperative design process, a group, which consists of some 
designers, is organized to implement each subsystem.
A group may divide the subsystem into more smaller blocks, each of which
is designed by a subgroup of the group.
After dividing the system in such a way, design of some blocks is
assigned to a designer, and he implements those blocks.

Since the subsystem which is assigned to a group uses the blocks
assigned to its members, the group cannot complete design of the
subsystem without the implementations of the blocks.
For this reason, his blocks are shared among the members of the group  
when he is convinced that he found and fixed all errors in the blocks.
The shared blocks are merged into an implementation of the subsystem,
and the members of the group continue design of the subsystem with
testing it a number of times.

When some errors are found in his implementation in the design process
of the subsystem,
it is prevented to access from the members of the group 
except him, and he tests and modifies it until the errors are
corrected.
When the group finishes correcting errors in the subsystem, 
it is shared among the members of a larger group, and design 
process proceeds similarly until whole the implementation of the system to
be designed would be completely implemented.

A lot of design data are created during these design processes; 
descriptions of the specifications of the subsystems, its
implementations, input data for tests, results of the tests, etc.
A designer
may refer to these design data in various ways: 
he may read other design data ---
whose owner may be other designer --- or its auxiliary data,
such as their names, the time of creation, the time of the latest
update, their owners etc; he may copy or snapshot other design data; he may test
the design data shared among other designers; he may create a new
version of the design data. 

In a design process, even the design data with errors may be accessed.
For example, they may be referred to correct similar errors in another
design data.
All the design data must be remained until the process finished, in
order to be able to refer them at any time during the process.
Moreover, there must be some kinds of distinctions between the design
data with no error and with some errors, in order to prevent from
designers' misuse of a design data with some errors.

We also notice that the referring data are in most cases under
implementation.
This means that design data which a designer is referring may be updated
frequently.
As a result, he may redesign his blocks with referring the new design data.

\subsection{Requirements for Database Systems to Support Cooperative
Design}
\label{subsec:require}
As described in Section \ref{sec:intro}, it has been very popular to
design circuits or softwares on CAD systems with databases in order to
manage a lot of design data efficiently.
There are various requirements for CAD databases to support
cooperative design on CAD systems.

First, varieties of transactions are necessary for the cooperative design processes 
on CAD systems, as follows:
\begin{itemize}
 \item reading design data%\ ({\em read})
 \item updating design data if it is allowed%\ ({\em write})
 \item reading the auxiliary data of designs%\ ({\em readinfo})
 \item updating the auxiliary data of designs%\ ({\em writeinfo})
 \item copying design data into a database%\ ({\em copy})
 \item making a snapshot of  design data%\ ({\em snapshot})
 \item making design data to be shared among the members who belongs a
same group%\ ({\em up})
 \item making shared design data not to be accessed from other designers
of a group%\ ({\em down})
 \item merging some design data to create a new design data%\ ({\em merge})
 \item testing design data with test tools and some test data%\ ({\em test}).
\end{itemize}
CAD database systems must provide these transactions.

Second, CAD database systems must have several 
functions to support these transactions.
\begin{enumerate}
 \item{\label{req:manage}} a function to manage various design data

Various design data are created on design processes on CAD systems, and
CAD database systems must manage these data well.
For example, CAD database systems must make some restrictions to the
transactions to design data with some errors.

 \item{\label{req:get}} a function to get all 
the information necessary for his design processes, including other 
designers' data and their auxiliary data.

In cooperative design, a designer frequently wants to refer others'
design data during his design process.
CAD database systems must provide some methods to support these
references.

 \item{\label{req:prevent}} a function to prevent other designers to 
access his design without his permission.

In contrast to the function \ref{req:get}, a designer has a requirement
of preventing other designers to read his design data, update them, or
copy them, etc, without his permission.

If it is allowed to read his design data including some errors, other
designers may read them and may design systems with errors.
If a designer's data under implementation would be updated without his
permission, he could not know what changes would be made in his design
or why those changes would be made.

 \item{\label{req:errornotify}} a function to notify the errors in his 
design data which other designers find.

If some errors are found in a designer's data which he does not allow
other designers to update, the errors must be notified to him to correct
them.

 \item{\label{req:changenotify}} a function to notify the changes of 
design data to the designers who have referred them.

A design data which was implemented with referring other data is
more or less influenced by the contents of the referred data.
If the contents of the referred data are changed, the changes must be 
notified to all the designers who have implemented systems with 
referring them.
\end{enumerate}

\subsection{Conventional Database Systems for Cooperative Design}
Conventional CAD database systems for cooperative design are 
constructed using a hierarchy of a public database, project databases, 
and private databases, as shown in Figure \ref{fig:groupdb}
\cite{Chou:vldb86,Klahold:sigmod85}.
\label{subsec:conventionaldb}
\begin{figure}
\begin{center}
\setlength{\unitlength}{0.0125in}%
\begin{picture}(360,297)(180,484)
\thicklines
\put(510,490){\oval( 60, 10)[bl]}
\put(510,490){\oval( 60, 10)[br]}
\put(510,530){\oval(60,10)}
\put(540,530){\line( 0,-1){ 40}}
\put(480,530){\line( 0,-1){ 40}}
\put(490,505){\makebox(0,0)[lb]{\raisebox{0pt}[0pt][0pt]{\twlrm personal}}}
\put(410,490){\oval( 60, 10)[bl]}
\put(410,490){\oval( 60, 10)[br]}
\put(410,530){\oval(60,10)}
\put(380,530){\line( 0,-1){ 40}}
\put(440,530){\line( 0,-1){ 40}}
\put(390,505){\makebox(0,0)[lb]{\raisebox{0pt}[0pt][0pt]{\twlrm personal}}}
\put(310,490){\oval( 60, 10)[bl]}
\put(310,490){\oval( 60, 10)[br]}
\put(310,530){\oval(60,10)}
\put(340,530){\line( 0,-1){ 40}}
\put(280,530){\line( 0,-1){ 40}}
\put(290,505){\makebox(0,0)[lb]{\raisebox{0pt}[0pt][0pt]{\twlrm personal}}}
\put(210,490){\oval( 60, 10)[bl]}
\put(210,490){\oval( 60, 10)[br]}
\put(210,530){\oval(60,10)}
\put(180,530){\line( 0,-1){ 40}}
\put(240,530){\line( 0,-1){ 40}}
\put(190,505){\makebox(0,0)[lb]{\raisebox{0pt}[0pt][0pt]{\twlrm personal}}}
\put(358,700){\oval(156, 20)[bl]}
\put(358,700){\oval(154, 20)[br]}
\put(360,770){\oval(152,22)}
\put(435,770){\line( 0,-1){ 70}}
\put(280,770){\line( 0,-1){ 70}}
\put(315,730){\makebox(0,0)[lb]{\raisebox{0pt}[0pt][0pt]{\twlrm common database}}}
\put(265,585){\oval( 90, 20)[bl]}
\put(265,585){\oval( 90, 20)[br]}
\put(265,635){\oval(92,22)}
\put(220,635){\line( 0,-1){ 50}}
\put(310,635){\line( 0,-1){ 50}}
\put(245,600){\makebox(0,0)[lb]{\raisebox{0pt}[0pt][0pt]{\twlrm group db}}}
\put(460,585){\oval( 90, 20)[bl]}
\put(460,585){\oval( 90, 20)[br]}
\put(460,635){\oval(92,22)}
\put(505,635){\line( 0,-1){ 50}}
\put(415,635){\line( 0,-1){ 50}}
\put(440,600){\makebox(0,0)[lb]{\raisebox{0pt}[0pt][0pt]{\twlrm group db}}}
\put(480,580){\vector(-2, 3){  0}}
\put(480,580){\vector( 2,-3){ 32.308}}
\put(440,580){\vector( 2, 3){  0}}
\put(440,580){\vector(-2,-3){ 32.308}}
\put(285,575){\vector(-1, 2){  0}}
\put(285,575){\vector( 1,-2){ 23}}
\put(245,575){\vector( 3, 4){  0}}
\put(245,575){\vector(-3,-4){ 34.200}}
\put(385,690){\vector(-4, 3){  0}}
\put(385,690){\vector( 4,-3){ 74.400}}
\put(335,690){\vector( 4, 3){  0}}
\put(335,690){\vector(-4,-3){ 71.200}}
\end{picture}

\end{center}
\caption{Conventional Database Model for Cooperative Design}
\label{fig:groupdb}
\end{figure}

The design data in the public database are accessible by any designer.
The design data in a private database are owned and accessed by only one
user.
A designer places design data in his group database when he wants to
share data with other designers who belong to the same group.
Other designers of the same group copy them, update them, etc, as new
versions in the group database.

Such a database system satisfies the requirements \ref{req:prevent} and
\ref{req:changenotify} in Section \ref{subsec:require}.
The design data which are placed in a designer's private database are
accessible only to himself, and he permits other designers in a same group
to update his design data by placing it in the group's database.
In \cite{Chou:vldb86}, message-based and flag-based change
notification mechanisms are also provided.

However, there are several problems in this database structure.
The first problem is that it does not always satisfy the requirement 
\ref{req:get}.
All the design data a designer can access are in the public database, his
private database, and the database of the group in which he belongs.
He cannot get any information from the design data on other databases.
Since it is difficult to divide a design object so that each subsystem
could be implemented independently, CAD database systems must support
more flexible data management methods for cooperative design that
conventional methods.

The second problem is that it does not provide any methods for the
requirement \ref{req:errornotify}.
The conventional model assumes that shared data in a group database have
no error and have no need for testing.
This assumption is invalid in actual cooperative design processes,
because there is an requirement to share design data with some errors,
for the reason that its errors cannot be fixed unless it is tested by
the group.

The third problem is that methods for the requirement
\ref{req:manage} are insufficient for actual cooperative design process.
The conventional model provides no method but placing design data with
some errors in a database, which can access only the members of a group.
This method is insufficient for the same reason as that of the second problem.
