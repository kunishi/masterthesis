\section{A Data Management Model for Cooperative Design}
\label{sec:dmm}
In this section, we describe several methods necessary for 
cooperative design on CAD systems,
summarized in Section \ref{subsec:coop-process}.
In the following sections, we suppose that a CAD database is 
constructed using a hierarchy
of a public database, project databases, and private databases, as
described in Section \ref{subsec:conventionaldb}.

\subsection{Preliminaries}
\label{subsec:prelim}
In this section, we formally define several concepts in cooperative design
processes.

%\begin{definition} \label{def:data}
%Define a design data $D$ as
%\[ D = (name, version). \] \hfill$\Box$
%\end{definition}

First we give a formal notation of designers to define a concept
``group".
\begin{definition} \label{def:users}
Define \[ U=\{u_{1}, u_{2}, \cdots, u_{m}\} \] as a set of designers, where
$u_i (1 \leq i \leq m)$ is a designer. \hfill$\Box$
\end{definition}

In cooperative design, a number of designers relates to one project.
Assumed that the members of each project are different from each other,
we can define a concept of a project member, ``group",  as follows:
\begin{definition} \label{def:groups}
Define \[ G = \{g_1, g_2, \cdots, g_n\} \] as a set of project groups,
where $g_i (1 \leq i \leq n)$ is a project group, which is a subset of
$U$. \hfill$\Box$
\end{definition}

\begin{definition} \label{def:parent}
If two groups $g$ and $g'$ satisfy the following conditions
\[ g' \in G \mbox{ s.t. } g \subseteq g', 
\mbox{ no group } g'' \mbox{ s.t. } g \subseteq g'' \subseteq g', \] 
we call $g'$ as a parent of $g$.\\
Define a child of group $g$ as
\begin{center}
$g'$ is a parent of $g$ $\iff$ $g$ is a child of $g'$.
\end{center}\hfill$\Box$
\end{definition}
In general, a group $g$ has a number of parents and children.

\begin{definition} \label{def:ancestor}
If two groups $g$ and $g'$ satisfy the following conditions
\[ g' \in G \mbox{ s.t. } g \subseteq g' ,\] 
we say $g'$ as an ancestor of $g$.\\
If a group $g'$ is an ancestor of a group $g$, we call $g$ as a
descendant of $g'$.  \hfill$\Box$
\end{definition}

\subsection{Data Management Based on Test Results}
\label{subsec:tfs-lattice}
\subsubsection{Management of Test Results}
\label{subsubsec:tfs-lattice}
In the design processes on CAD systems, many tools are used for tests,
%and the design data which passes tests of more test tools may be
%considered to have more validity.
and the rank of validity for a design data can be defined by a set of
test tools whose tests are passed.
In the case of cooperative design, a designer is likely to decide 
whether to share a design data by what kinds of tests it has passed, 
so the information about test results is much important for design on 
CAD systems.

A designer often refers results of the latest test to a design data in 
order to decide whether to correct it. 
If the latest test was failed, he must investigate the causes of the
errors and correct them.

However, results of the latest test is not enough to decide whether 
to share design data. 
He cannot make an exact judge from the results whether there are no more 
errors in it.
For example, when design data passed design rule checking, which is the
latest test, he convinces it to have no error against design rules.
However, its results are insufficient to decide whether to share it as
an released data, since he cannot get any information about other kinds of
errors. 
Thus results of other tests are needed.

For this reason, we provide flags, called {\em test flags}, for each 
design data for the number of test tools.
Each flag $f_i$ means that the latest test using a test tool $t_i$ is 
passed or failed.
In general, each design data has different test flags on each other,
since test tools applicable to each design data may be different from
each other.
Design data which are the same but in different databases may have
different test flags. 

There are $n$ test flags on a design data, which form a {\em test-flag
vector} $tfs = (f_1, f_2, \cdots, f_n)$, if CAD systems has $n$ test
tools $t_1, t_2, \cdots, t_n$.
A set of all the values of \tfs equals to $\{0,1\}^n$, and 
organizes a lattice, called {\em tfs-lattice}, which is constructed as
follows:
\begin{quote}
$glb(tfs_1, tfs_2) = tfs_1 \wedge tfs_2$\\
$lub(tfs_1, tfs_2) = tfs_1 \vee tfs_2$\\
where $\wedge$ and $\vee$ means bitwise-and and bitwise-or, respectively.
\end{quote}
An example of tfs-lattice is shown in Figure \ref{fig:tfs-lattice}. 
\begin{figure}
\begin{center}
\setlength{\unitlength}{0.0125in}%
\begin{picture}(220,231)(200,515)
\thicklines
\put(420,660){\line( 0,-1){ 50}}
\put(215,660){\line( 0,-1){ 50}}
\put(315,730){\line( 0,-1){ 50}}
\put(315,660){\line(-2,-1){100}}
\put(315,660){\line( 2,-1){100}}
\put(415,590){\line(-5,-3){ 97.794}}
\put(215,590){\line( 5,-3){ 97.794}}
\put(415,660){\line(-2,-1){100}}
\put(315,730){\line( 2,-1){100}}
\put(315,590){\line( 0,-1){ 55}}
\put(215,660){\line( 2,-1){100}}
\put(315,730){\line(-2,-1){100}}
\put(400,595){\makebox(0,0)[lb]{\raisebox{0pt}[0pt][0pt]{\twlrm 1, 0, 0}}}
\put(300,595){\makebox(0,0)[lb]{\raisebox{0pt}[0pt][0pt]{\twlrm 0, 1, 0}}}
\put(400,665){\makebox(0,0)[lb]{\raisebox{0pt}[0pt][0pt]{\twlrm 1, 1, 0}}}
\put(300,515){\makebox(0,0)[lb]{\raisebox{0pt}[0pt][0pt]{\twlrm 0, 0, 0}}}
\put(200,595){\makebox(0,0)[lb]{\raisebox{0pt}[0pt][0pt]{\twlrm 0, 0, 1}}}
\put(200,665){\makebox(0,0)[lb]{\raisebox{0pt}[0pt][0pt]{\twlrm 0, 1, 1}}}
\put(300,665){\makebox(0,0)[lb]{\raisebox{0pt}[0pt][0pt]{\twlrm 1, 0, 1}}}
\put(300,735){\makebox(0,0)[lb]{\raisebox{0pt}[0pt][0pt]{\twlrm 1, 1, 1}}}
\end{picture}

\\Each node of the lattice is a test-flag vector which means that: 
\[ (simulation1, simulation2, design rule check) \]
\caption{Lattice of Test-Flags Vectors}
\label{fig:tfs-lattice}
\end{center}
\end{figure}

Using \tfs, more delicate data management can be done.
For example, it can manage that only design data which passed tests with a 
certain number of tools are made to be sharable; 
or it also can do a management to make some kinds of tests to 
a design data to be valid only for the case that it passed a certain 
kinds of tests.

In general, it is possible to manage transactions on design
data using a function $F(u, t, db, f_1, f_2, \cdots, f_n)$,
where $u$ is a designer, $t$ is a transaction, $db$ is a database in which
the data places, and $f_i$ is a test flag. 
$F$ equals to 1 iff (if and only if) the transaction is valid.
In the previous examples, $F$ can be defined as a threshold function about
the variables $f_1, f_2, \cdots, f_n$, and a
function that depends on only one variable, respectively.
Hereafter we call $F$ as an {\em access control function}.

\subsubsection{Dependencies among Test Tools}
\label{subsubsec:testtooldep}
In this section, we describe a management method on tfs-lattice for 
transactions.

We observe one characteristic about test tools on CAD systems.
There are some cases that they have dependencies on each other.
Examples are as follows:

\begin{example}\label{ex:rulecheck}
Logic circuits which passed the tests of a logic simulator are
expected to pass the tests of a design rule checker, since circuits
which does not satisfy design rules cannot pass logic simulation.
Then a designer might omit a design rule check to the circuits which
passed logic simulation.\hfill$\Box$
\end{example}

\begin{example}\label{ex:timesymbol}
It is reported in \cite{Ishiura:Jousho90} that in some logic circuits, 
time-symbolic simulation gives more exact results than min/max delay 
simulation.
This also means that logic circuits which passed tests of the min/max delay
simulation always pass tests of the time-symbolic simulation.\hfill$\Box$
\end{example}

We formalize such dependencies among test tools to provide a concept 
``\testtool dependency".
\begin{definition}\label{def:testtool-depend}
If all the design data which passed tests using a tool $t_1$ are 
expected to pass tests of another tool $t_2$,
we would say ``there is a \testtool dependency from $t_1$ to $t_2$", and
notate as:
\[ t_1 \testtooldep t_2 .\]\hfill$\Box$
\end{definition}

\begin{example}
In the previous examples \ref{ex:rulecheck} and \ref{ex:timesymbol}, 
there are \testtool dependencies 
\begin{quote}
logic-simulator $\testtooldep$ design-rule-checker,\\
min-max-delay-sim $\testtooldep$ time-symbolic-sim,
\end{quote} respectively.\hfill$\Box$
\end{example}

If a test for a design data using test tool $t_i$ has passed, value of 
a flag $f_i$ becomes 1, and, a value of \tfs changes  the bitwise-or 
between \tfs and the vector of length $n$ that all but the $i$-th 
flag is 0.
However, when test tool $t_i$ has a \testtool dependency to another 
test tool $t_j$,
it can be considered that design data which passed the test 
of $t_i$ would pass the test of $t_j$.
In that case, test of $t_j$ can be omitted, and design data whose $f_i$
is 1 but $f_j$ is 0 can be handled as if $f_i$ and $f_j$ are both 1.
It is formalized by an access control function $F$ in Section 
\ref{subsec:tfs-lattice} as follows.
If \testtool dependency $t_i \testtooldep t_j$ exists, then
\[ F\left|_{f_i=1,f_j=0} \right. \equiv F\left|_{f_i=1, f_j=1} \right. .\]

Though a \testtool dependency $t_1 \testtooldep t_2$ exists, there are
cases that tests of $t_2$ are required.
One of the cases is that a designer uses $t_2$ and corrects as many
errors as possible, to make tests of $t_1$ to be more simple.

\subsection{Data Management after Testing}
 \label{subsec:aftertest}
There are two different purposes to test design data on CAD systems:
\begin{enumerate}
\item{\label{confirm}} to confirm that the design data has no error,
\item{\label{specify}} to specify the causes of the errors.
\end{enumerate}
In the case \ref{confirm}, the design data's validity has 
increased if the test is passed.
On the other hand, there are some errors in the design data in the
case \ref{specify}, and it has less validity than it had before
the errors were found.
We must distinct those two cases, since design data's validity is 
an important factor for deciding whether to access transactions on it.

We can distinct those two cases by results of the latest test.
The case \ref{confirm} occurs only when design data passed the latest 
test; on the other hand, the case \ref{specify} occurs only when the
latest test was failed and causes of the errors have not been specified yet.

Then we provide a flag called \deterr flag for each design data. 
If a designer has tested his design data to find some errors in its
results, the value of \deterr of the design data becomes
1.
And if he finished specifying causes of the errors, the value of 
\deterr becomes 0.
In other words, if this flag equals to 1, the owner of the data is now 
specifying causes of the errors.

Using this flag, CAD database can distinct the cases \ref{confirm} and
\ref{specify}, and can manage the same transaction in a different way.
For example, it can do a management to make the design data to be 
accessible until the causes of some errors in it are specified.

This management method is formalized as a function\\ $F'(u, t,
db, f_1,
f_2, \cdots, f_n, detectError)$, an extension of the access control
function $F$ in Section
\ref{subsec:tfs-lattice}.
In the above example, $F'$ is defined as $F' = F \vee detectError$, 
where $F$ is a function in Section \ref{subsec:tfs-lattice}.

\subsection{A Management Model of Data Sharing}
 \label{subsec:updown}
On a database hierarchy as mentioned in Section
\ref{subsec:conventionaldb}, each group has its own group database,
in which design data are placed to be shared among the members of the 
group, and other designers cannot write into the group database.

Based on these facts, CAD database can support data sharing on a
database hierarchy only to provide these two transactions: placing design
data into a parent database, and placing design data into a child
database.
Then we provide two transactions in order to manage
data sharing on the database hierarchy.
\begin{enumerate}
 \item \up transaction\\
a transaction to place a design data into the parent database,
 \item \down transaction\\
a transaction to place a design data into the child database for the
data.
\end{enumerate}

Here we use concepts a ``child database" of a database for a design
data in accordance with the following definition.
\begin{definition}\label{def:child-for-dd}
Define a child database $child(db)$ of database $db$ for a design 
data $d$ as a child database of $db$ which is an ancestor of the 
designer of $d$. \hfill$\Box$
\end{definition}

\up and \down transactions have some properties:
\begin{itemize}
\item \up transaction on a design data is valid only when a group which 
owned the data regards it to be designed completely.
\item When a \down transaction is executed on a design data, the owner
of the data is changed to the one who owned it before it was upped.
\end{itemize}

Using the management methods described in Sections
\ref{subsec:tfs-lattice} and \ref{subsec:aftertest}, formalization of 
management methods about \up and \down transactions 
satisfying these properties is as follows.

A management method of \up transaction is as follows.
\up transaction of a designer $u$ will be accepted only if $F(u, up, db, f_1, f_2, 
\cdots, f_n) = 1 \wedge detectError = 0$ and he has an authority to 
write the parent database.

\down transaction is managed as follows.
\down transaction only runs when \deterr flag equals to 1.
It is accepted only if a designer who executed the \down transaction
has an authority to write the child database for the downed data.
A value of \tfs is set to the one when it was upped.

\subsection{An Example of the Data Management Model}
 \label{subsec:example}
In this section, we show an example of a data management model based on
methods described in Sections \ref{subsec:tfs-lattice},
\ref{subsec:aftertest}, and \ref{subsec:updown}.

\subsubsection{A User Transaction Model for Cooperative Design}
 \label{subsubsec:transmodel}
In this example, we consider a user transaction model for cooperative
design as follows:
\begin{itemize}
\item read($D$)\\
read design data $D$.
\item update($D$)\\
update design data $D$, not creating a new version of $D$.
\item create($D$, $V$, $DB$)\\
create a new version $V$ of design data $D$ into database $DB$.
\item readinfo($D$)\\
read auxiliary data of design data $D$.
\item writeinfo($D$)\\
rewrite auxiliary data of design data $D$.
\item copy($D$, $V$, $DB$)\\
copy design data $D$ into database $DB$ as a version $V$.
\item snapshot($D$, $DB$)\\
make a snapshot of design data $D$ into database $DB$.
\item up($D$)\\
make design data $D$ to be shared among the members of the parent 
group.
\item down($D$)\\
make design data $D$ to be shared among only the members of the
child for data $D$
\item merge($D_1$, $D_2$, $\cdots$, $D_n$, $D$)\\
merge design data $D_1, D_2, \cdots, D_n$ as a design data $D$.
\item test($TOOL$, $D$, $TDATA$)\\
test design data $D$ with a test tool $TOOL$ using test data $TDATA$.
\end{itemize}

\subsubsection{A Data Management Model}
 \label{subsubsec:construct}
In this section, we show an example of a data management model for the
user transaction model described in Section \ref{subsubsec:transmodel},
using management methods in Sections \ref{subsec:tfs-lattice},
\ref{subsec:aftertest}, and \ref{subsec:updown}.

First, the following user transactions are modeled without the
management methods described in above sections.
\begin{itemize}
\item readinfo($D$)\\
This user transaction is always accepted for all designers.
\item writeinfo($D$)\\
This user transaction is accepted only if it is executed by the owner of
$D$.
\end{itemize}

Second, several user transactions are divided into some 
transactions as follows:
\begin{itemize}
\item copy($D$, $V$, $DB$)\\
This user transaction is divided into some subtransactions:
read($D$), create($D$, $V_0$, $DB$).
It is accepted only when both of them are accepted.
\item snapshot($D$, $DB$)\\
This user transaction is also divided into some subtransactions:
read($D$) and create($D$, $V_0$, $DB$).
It is accepted only when both of them are accepted.
\item up($D$)\\
This user transaction is divided into some subtransactions:
read($D$) and create($D$, $V_0$, $parent(DB)$), where $parent(DB)$ 
is the parent database of the database in which $D$ is placed.
It is accepted only when both of them are accepted and \deterr of $D$ 
equals to 0.

When it is finished, the owner of $D$ is changed to a group who owns
$parent(DB)$.
\item down($D$)\\
This user transaction is divided into some subtransactions:
read($D$) and create($D$, $V_0$, $child(DB, D)$), where $child(DB, 
D)$ is the child database for the data $D$.
It is accepted only when both of them are accepted.

When it is finished, the owner of $D$ is changed to a group who owns
$child(DB, D)$.
\item merge($D_1$, $D_2$, $\cdots$, $D_n$, $D$)\\
This user transaction is divided into some subtransactions:
read($D_1$), read($D_2$), $\cdots$, read($D_n$), and create($D, V_0, DB$), 
where $DB$ is a database in which $D_1, D_2, \cdots, D_n$ are placed.
It is accepted only when all of them are accepted.
\item test($TOOL$, $D$, $TDATA$)\\
This user transaction is divided into some subtransactions:
read($D$), read($TDATA$), and create($r, V_0, DB$), where
$r$ is results of the test.
It is accepted only when all of them are accepted.

If the test would be passed, test flag of $D$ which corresponds to 
$TOOL$ is changed to 1. If failed, the test flag is changed to 0, and 
\deterr flag becomes 1. 
\end{itemize}

Next, we model the transactions read($D$), update($D$), and create($D$, $V$,
$DB$) with a function $F'$ in \ref{subsec:aftertest}.
Here we model them according to the following principles.
\begin{itemize}
\item read($D$) is acceptable for designers who don't belong to the
group owned $DB$ as well as the members of the group.
\item create($D$, $V$, $DB$) is only acceptable for the members of the 
group owned $DB$.
\item update($D$) is only acceptable for a few of the members of the
group owned $DB$.
\end{itemize}
Reasons are:
\begin{enumerate}
\item{\label{reason:req}} There may be a requirement that a designer 
implements his design with referring other groups' design data.
\item In the case \ref{reason:req}, update of shared design data may affect
a large number of design data.
Then there must be more authorities to update shared design data. 
\end{enumerate}

We construct a function $F'$ as 
\[ F' = F''(u, t, db, evalTest(f_1, f_2, \cdots, f_n), detectError). \]
That is, $f_1, f_2, \cdots, f_n$ are made to be independent of other 
variables.

Details of $F''$ is
\begin{itemize}
\item $F''(u, read(D), DB, evalTest, 0) = 1$ for all designer $u$\\
$F''(u, read(D), DB, evalTest, 1) = 1$ iff $u$ is a member of group($DB$)
\item $F''(u, create(D, V, DB), DB, evalTest, detectError) = 1$ iff $u$ is a 
member of $group(DB)$
\item $F''(u, update(D), DB, evalTest, 0) = 1$ iff $u$ is a member of
group($DB$)\\
$F''(u, update(D), DB, evalTest, 1) = 1$ iff $u$ is a member of
group($child(DB, D)$) 
\end{itemize}

At last, a function $evalTest$ would be modeled.
Suppose the CAD system has test tools $t_1, t_2, t_3$, $evalTest$ is a
Boolean function of $f_1, f_2, f_3$.
For example, in the case that design data must pass tests
of at least two tools to be shared.
%(An example of this situation is Example \ref{ex:rulecheck} in Section
%\ref{subsubsec:testtooldep}.)
$evalTest$ is formed as a threshold function
\[ evalTest(f_1, f_2, f_3) = f_1f_2 + f_2f_3 + f_3f_1 .\]
If there is a \testtool dependency $t_1 \testtooldep t_2$, then
$evalTest$ is 
\[ evalTest(f_1, f_2, f_3) = f_1 + f_2f_3 + f_3f_1 .\]
