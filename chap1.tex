\section{Introduction}
\label{sec:intro}
In order to design large and complex hardware and software, use of CAD 
(Computer-Aided Design) systems is essential.
Due to the recent requirement to realize large systems, database
function must be combined with CAD systems.
%In the fields of circuit design or software design, in recent years,
%design objects have been so larger and more complicated that they could
%hardly be designed by hand.
%Then it has been much popular to design circuits or
%softwares on Computer-Aided Design (CAD) systems.

In a design process utilizing a CAD system, many design data are created by
designers or the CAD system; the descriptions of the design objects,
their auxiliary data (their creation date etc.), input data for tests, 
and the results of the tests, etc.
Since these data are often referred to at any time of the design 
process, they must be retained until the design is completed.
For this reason, database systems are used in CAD systems in order to 
manage design data efficiently, and there are many papers on CAD 
databases
\cite{Batory:acmtrans85,Bancilhon:vldb85,Chou:vldb86,Katz:da86,Ketabchi:ieeetransse88,Klahold:sigmod85}.

CAD database systems must have new functions which are not 
required for traditional database systems.
One of these functions is to support cooperative design.
Large and complicated design object, which could hardly
be designed by one designer, are usually implemented cooperatively by a
number of designers.
CAD database systems must offer some functions which lighten the
designers' burden for cooperative design, such as the methods of 
sharing design data, the control methods for accessing design data, 
change notification methods, etc.

A conventional design database system gives a management method for cooperative
design based on a hierarchy of databases
\cite{Chou:vldb86,Klahold:sigmod85}.
The system provides a public database for all the designers, a group
database for each group on the design process, and a private database
for each designer.
Design data which are shared among the members of a project are placed 
in the database of the project.
Any of the members can execute any transactions to the shared data, and
the other members cannot execute any transactions to them.
If a design data is assumed to be error free, it is placed into the
parent database in the hierarchy.
On the other hand, if some errors are found in a design data, it is 
placed back into a child database in the hierarchy.

This management method has some merits to support cooperative design.
First, it provides simple and powerful data sharing methods and 
transaction control methods.
Second, it is suitable for distributed design environment, which has
been popular with the spread of low-priced and powerful
workstations.

However, this management method has less flexibility of controlling
transactions.
For example, it often happens that shared data in a group database
can be read by any member of the group, but few of the members are
allowed to update them.
The conventional method cannot support such a transaction control, since 
permission to access design data depends on only the 
database the design data places, not on the kind of transactions.

In this thesis, we propose a new data management model on the database
hierarchy.
First we define a concept of an {\em access control function}, which 
determines whether to permit a transaction to operate on a design data.
It is defined for each design data as a function from $U$ (a set of 
users), $T$ (a set of transactions), $DB$ (a set of databases), and 
an $n$-bit vector of {\em test flags}, each of 
which is provided for a test tool on CAD systems and reflects the 
result of the latest test using the tool.
We also show an extension of the function in order to distinct two
situations of testing: confirming its validity and specifying causes of
errors.

Next we propose a concept of ``\testtool {\em dependency}", a binary 
dependency between two test tools whose test results always implys the others.
We also mention the relationships between the access control function
and this dependency.

We also show an example of the access control function and the \testtool
dependency under a user transaction model, which is similar to actual 
user transactions on CAD databases.

Finally we consider several extensions of the access control function
and the \testtool dependency.
We also consider concepts of user hierarchy, change notification
methods, and error notification methods on our data management model.

The remainder of this thesis is organized as follows.
In Section \ref{sec:cooperativedb} we summarize the cooperative design
process on CAD systems.
We also mention the conventional CAD database systems for cooperative 
design, and consider problems of such database systems in this section.
We describe our data management model, the access control function and
the \testtool dependency in Section \ref{sec:dmm}.
In Section \ref{sec:extension}, extensions of our data management model are
discussed.
Section \ref{sec:conclu} is concluding remarks.
