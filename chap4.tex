\section{Extensions of the Management Model for Cooperative Design}
\label{sec:extension}
In this section, we will discuss several extensions of our data management
model described in Section \ref{sec:dmm}.

\subsection{Extensions of Dependencies among Test Tools}
\label{subsec:ext-testtooldep}
We proposed \testtool dependencies, dependencies among test tools,
in Section \ref{subsubsec:testtooldep}. 
The meaning of a \testtool dependency $t_i \testtooldep t_j$ is that 
all the design data which passed tests using a tool $t_i$ are expected to
pass tests of another tool $t_j$, by Definition
\ref{def:testtool-depend}.

There are many ways of extensions for this dependency.
First, we can consider more dependencies among test tools on CAD
systems.
\begin{itemize}
\item If a design data passed a test of tool $t_i$, it can be
tested by another test tool $t_j$.
\item If a design data passed a test of tool $t_i$, it cannot be 
tested by another test tool $t_j$.
\end{itemize}
Second, we can extend the dependencies, which is binary, to the non-binary
ones between two sets of test tools.

Then we extend the \testtool dependency as follows:
\begin{definition}
\label{def:ext-testtooldep}
Define an {\em extended testTool dependency} as
\[ t_{i_1}t_{i_2}\cdots t_{i_n} \longrightarrow_{tool(kind)} 
t_{j_1}t_{j_2}\cdots t_{j_m} \]
where $t_{i_1}, t_{i_2}, \cdots, t_{i_n}, t_{j_1}, t_{j_2}, \cdots,
t_{j_m}$ are test tools, and $kind$ is one of the following commands.
\begin{itemize}
\item $omit$\\
tests of $t_{j_1}, t_{j_2}, \cdots, t_{j_m}$ can be omitted if tests of
$t_{i_1}, t_{i_2}, \cdots, t_{i_n}$ are all passed.
\item $permit$\\
tests of $t_{j_1}, t_{j_2}, \cdots, t_{j_m}$ are permitted if tests of
$t_{i_1}, t_{i_2}, \cdots, t_{i_n}$ are all passed.
\item $prohibit$\\
tests of $t_{j_1}, t_{j_2}, \cdots, t_{j_m}$ are prohibited if tests
of $t_{i_1}, t_{i_2}, \cdots, t_{i_n}$ are all passed.
\end{itemize}\hfill$\Box$
\end{definition}

These dependencies are written by the access control function $F$ in Section 
\ref{subsubsec:testtooldep} as:
\begin{itemize}
\item the case of $kind = omit$\\
It is as same as in Section \ref{subsubsec:testtooldep}.
\item the case of $kind = permit$\\
For each $t$ in $\{t_{j_1}, t_{j_2}, \cdots, t_{j_m}\}$, 
\[ F(u, t, db, f_1, f_2, \cdots, f_n) = 1 \iff f_{i_1} \wedge f_{i_2} 
\wedge \cdots \wedge f_{i_n} = 1. \]
\item the case of $kind = prohibit$\\
For each $t$ in $\{t_{j_1}, t_{j_2}, \cdots, t_{j_m}\}$, 
\[ F(u, t, db, f_1, f_2, \cdots, f_n) = 0 \iff f_{i_1} \wedge f_{i_2} 
\wedge \cdots \wedge f_{i_n} = 1. \]
\end{itemize}

\subsection{Design Process Automaton} \label{subsec:dpa}
Data management by the access control function $F$ in Section 
\ref{subsubsec:testtooldep} is based on results of the latest test of 
each test tools.
Then it cannot support such a case that design data can be shared if it
passes tests of a tool more than $n$ times.
This situation often occurs in a design process of CAD systems since
designers repeats tests of a test tool with various input data.
In order to manage such cases, we define a design process automaton for 
each design data.
\begin{definition} \label{def:dpa}
Define a {\em design process automaton} ({\em DPA}) for each design data
as \[ DPA = (I, S, S_0, \delta) \]
where $I$, $S$ and $S_0$ are finite, nonempty sets of inputs, states,
and initial states, respectively; 
$\delta: I \times S \rightarrow S$ is the state transition function.
\hfill$\Box$\end{definition}

Figure \ref{fig:dpa} shows an example of {\em DPA} of a design data.
This {\em DPA} manages a design process that the design data must pass
tests using $t_1$ more than three times to be shared, but it 
may pass a test using $t_2$ only once.
\begin{figure}
%\begin{center}
\setlength{\unitlength}{0.0125in}%
\begin{picture}(433,110)(59,660)
\thicklines
\put(453,679){\vector( 0, 1){0}}
\put(472,679){\oval( 38, 30)[bl]}
\put(472,684){\oval( 38, 40)[br]}
\put(472,684){\oval( 38, 38)[tr]}
\put(472,688){\oval( 38, 30)[tl]}
\put(106,702){\oval( 16, 18)[br]}
\put(106,693){\vector( 0, 1){0}}
\put(434,683){\circle{20}}
\put(434,683){\circle{34}}
\put(342,753){\circle{34}}
\put(217,749){\circle{34}}
\put(119,683){\circle{34}}
\put(133,679){\vector( 1, 0){283}}
\put( 59,683){\vector( 1, 0){ 41}}
\put(416,688){\vector(-1, 0){278}}
\put(328,744){\vector(-4,-1){196.706}}
\put(127,697){\vector( 3, 2){ 70.846}}
\put(356,744){\vector( 4,-3){ 66.080}}
\put(235,749){\vector( 1, 0){ 88}}
\put(203,739){\vector(-3,-2){ 69.692}}
\put(491,660){\makebox(0,0)[lb]{\raisebox{0pt}[0pt][0pt]{\twlrm Pass(*)}}}
\put(249,665){\makebox(0,0)[lb]{\raisebox{0pt}[0pt][0pt]{\twlrm Pass($t_2$)}}}
\put(195,722){\makebox(0,0)[lb]{\raisebox{0pt}[0pt][0pt]{\twlrm Fail($t_1$)}}}
\put(268,716){\makebox(0,0)[lb]{\raisebox{0pt}[0pt][0pt]{\twlrm Fail($t_1$)}}}
\put(275,693){\makebox(0,0)[lb]{\raisebox{0pt}[0pt][0pt]{\twlrm Fail(*)}}}
\put(374,735){\makebox(0,0)[lb]{\raisebox{0pt}[0pt][0pt]{\twlrm Pass($t_1$)}}}
\put(244,753){\makebox(0,0)[lb]{\raisebox{0pt}[0pt][0pt]{\twlrm Pass($t_1$)}}}
\put(119,721){\makebox(0,0)[lb]{\raisebox{0pt}[0pt][0pt]{\twlrm Pass($t_1$)}}}
\put( 73,735){\makebox(0,0)[lb]{\raisebox{0pt}[0pt][0pt]{\twlrm Fail(*)}}}
\end{picture}

\caption{An Example of a Design Process Automaton}
\label{fig:dpa}
%\end{center}
\end{figure}

We can construct $F$ by {\em DFA} with the following $S$, $I$, $S_0$ 
and $\delta$:
\begin{itemize}
\item $S$ is a set of \tfs $(f_1, f_2, \cdots, f_n)$.
\item $I$ is a set \[ \{Pass(t), Fail(t) | t \in T, T \mbox{ is a set of
all test tools } \} .\] 
If $F(u, t, db, f_1, f_2, \cdots, f_n) = 1$, then {\em DPA} makes a 
transition by $Pass(t)$.
If $F(u, t, db, f_1, f_2, \cdots, f_n) = 0$, then it makes a transition
by $Fail(t)$.
\item $S_0$ is a set of \tfs,  $\{(0, 0, \cdots, 0)\}$.
\item $\delta$ is defined as
\[ \delta((f_1, f_2, \cdots, f_i, \cdots, f_n), Pass(t_i)) = (f_1, f_2, 
\cdots, 1, \cdots, f_n)\]
\[\delta((f_1, f_2, \cdots, f_i, \cdots, f_n), Fail(t_i)) = (f_1, f_2, 
\cdots, 0, \cdots, f_n)\]
\end{itemize}

\subsection{User Hierarchy}
\label{subsec:userhierarchy}
In both the conventional management model and our model in Section 
\ref{sec:dmm}, the members of a
group are modeled to have the same authority over design data.
In actual cooperative design, however, they may have different
authorities on each other.
For example, each group usually has a leader 
as a manager of the project of the group.
The leader usually has more authority over design data than normal
designers in order to do smooth managements.
He may be able to read some design data which is not 
accessible for normal designers, such as the ones in a private database
of one of the members.

Then we consider the way to manage such situations in this section.
First we define orderings between the members of a group.
\begin{definition} \label{def:user-part-order}
For each users $u_i$ and $u_j$ in the group, 
\[ u_i \succeq u_j \iff F(u_i, t, db, f_1, 
\cdots, f_n) \geq F(u_j, t, db, f_1, \cdots, f_n) \]
for all transactions $t$, 
where $F$ is the access control function of a design data $D$. 

In particular, 
\[ u_i = u_j \iff F(u_i, t, db, f_1, \cdots, f_n) =
F(u_j, t, db, f_1, \cdots, f_n) \]
for all $t$. \hfill$\Box$
\end{definition}
This ordering is partial, since a set of executable transactions of each
designer may be different from each other.
Then a set of all members of the group forms a hierarchy.
We call it as a {\em user hierarchy}.

We find a simple example of user hierarchy in an example in 
Section \ref{subsec:example}.
Since $F''(u, read(D), DB, evalTest, 1) = 1$ iff $u$ is a member of
group($DB$), $u_j \succeq u_k$, where $u_j$ is a member of $group(DB)$
while $u_k$ is a designer without group($DB$).
On the other hand, since $F''(u, update(D), DB, evalTest, 1) = 1$ iff
$u$ is a member of group($child(DB, D)$), $u_i \succeq u_j$, where
$u_i$ is a member of group($child(DB, D)$) while $u_j$ is a member of
group($DB$).
Then $u_i \succeq u_j \succeq u_k$.

\subsection{Change Notification}
 \label{subsec:notification}
As described in Section \ref{sec:cooperativedb}, changes of a
design data may affect the validity of other design data in 
cooperative design.
So notification of the changes is very important.

To consider methods for change notification, we find a problem to whom
the changes are notified.
Here we take an approach to notify the changes to the designer who 
referred the changed data.
The reason is that CAD systems cannot decide precisely whether to 
correct the affected data or not, and the designer must make some decisions.

In order to lighten designers' burden for searching the affected
data, the change notification must include information which 
data is affected.
Then CAD system must manage information which data may be
affected by the change of a design data.
We propose a concept of {\em referred data} for a design data.
\begin{definition} \label{def:referreddata}
Define a set of {\em referred data} for a design data $D$ as:
\begin{enumerate}
\item{\label{rd:copyD}} All design data which are created by transactions 
{\em create}, {\em copy}, {\em merge}, and {\em test} to $D$ are 
referred data.
\item{\label{rd:readD}} All design data which are designed using results 
of transactions {\em read} and {\em readinfo} to $D$ are referred data.
\item{\label{rd:copyrd}} All design data which are created by transactions 
{\em create}, {\em copy}, {\em merge}, and {\em test} to a referred 
data for $D$ are referred data.
\item{\label{rd:readrd}} All design data which are designed using results 
of transactions {\em read} and {\em readinfo} to a referred data for 
$D$ are referred data.
\end{enumerate}
\hfill$\Box$ \end{definition}
We notice that, in the cases \ref{rd:readD} and \ref{rd:readrd}, referred
data cannot be decided automatically, because a designer may implement 
some design objects simultaneously.

A design data which is a referred data for $D$ may be affected by the
change of $D$.
Since design data which had referred the changed data may even be affected,
we provide a method of notifying all referred data for
$D$ when $D$ is changed. 

%So we provide a \refhis for each design data, and holds a history 
%about all transactions which have or had referred the data.
%
We provide an {\em affected event queue} for each design data.
If a design data $D$ have changed, the system sends a message, which
includes information which data has changed, to all the referred data 
for $D$.
The message appends to the affected event queue of $D$.
It doesn't send to the owner of $D$ directly, because the changes may affect 
to older versions, which has no need to update.

When a designer $u$ accesses the affected data $AD$, all messages in its affected
event queue would be managed, the results of the management would
be sent to the owner of $AD$, and the owner would correct $AD$ 
according to the received results.
On the other hand, the system sends to $u$ a message to keep his transaction
waiting.

\subsection{Error Notification}
 \label{subsec:errornot}
In our data management model, in contrast with the conventional
approach, a designer $u$ can test a design data if the access control
function $F(u, test, db, f_1, \cdots, f_n) = 1$, even in the case he
doesn't belong to a group which owns it.
If he cannot update the design data in which he found some errors, CAD
system must provide a method to notify these errors to designers who can
update it.

If $F(u, test, db, f_1, \cdots, f_n) = 1 \wedge F(u, create, db, f_1,
\cdots, f_n) = 0 \wedge F(u, update, db, f_1, \cdots, f_n) = 0$, the 
system sends a message to the owner of the tested data what kind of 
tests were executed, and what the results were. 
