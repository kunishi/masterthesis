\section{Conclusion}
\label{sec:conclu}
In this thesis, a new data management model for cooperative
design has been presented.
It is based on a database hierarchy which is used in conventional data
management models, but it has more flexible management methods than the
conventional ones.

First an {\em access control function} has been defined, which manages several
user transactions in a flexible way mainly by results of the latest
tests of each test tools on CAD systems.
We have also shown an extension of the function in order to manage data
considering meanings of tests.

Next we have proposed a concept of \testtool {\em dependency}, a binary
dependency between two test tools whose test results always implys the
others.
We have also mentioned the relationships between the access control function
and this dependency.

Finally several extensions of the access control function
and the \testtool dependency have been discussed.
We have also considered a concept of user hierarchy and change notification
methods on our data management model.

In future we will research more characteristics about the access control
function, the \testtool dependencies, and their extensions.
In particular, we will research the ways to construct them for various 
requirements on CAD systems, and the mathematical properties on them.

We will also research how to support constructing our model.
Our model has more flexibility, but it is more difficult to organize
them into data management systems of some CAD systems.
Researches about the way of support this organization process, 
particularly about user interfaces, are important for its practical use
to CAD database systems.
