\begin{abstract}
In a design process utilizing a Computer-Aided Design (CAD) system, many 
design data are created by 
designers or the CAD system; the descriptions of the design objects,
their auxiliary data (their creation date etc.), input data for tests, 
and the results of the tests, etc.
Since these data are often referred to at any time of the design 
process, they must be retained until the design is completed.
For this reason, database systems must be combined with CAD systems 
in order to manage design data efficiently, and there are many 
papers on CAD databases.

CAD database systems must have new functions which are not 
required for traditional database systems.
One of these functions is to support cooperative design.
So large and complicated design object that it could hardly
be designed by one designer is usually implemented cooperatively by a
number of designers.
CAD database systems must offer some functions which make the
cooperation of the designers to be easy, such as the methods of 
sharing design data, the control methods for accessing design data, 
change notification methods, etc.

A conventional design database system gives a management method for 
cooperative design based on a hierarchy of databases.
However, this management method has less flexibility of controlling
transactions.

In this thesis, we propose a new data management model in the database
hierarchy.
First we define a concept of an {\em access control function}, which 
determines whether to permit a transaction to operate on a design data.
It is defined for each design data as a function from $U$ (a set of 
users), $T$ (a set of transactions), $DB$ (a set of databases), and 
an $n$-bit vector of {\em test flags}, each of 
which is provided for each test tool on CAD systems and reflects the 
result of the latest test of the test tools.
We also show an extension of the function in order to distinguish two
situations of testing: confirming its validity and specifying causes of
errors.

Next we propose a concept of ``\testtool {\em dependency}", a binary 
dependency between two test tools whose test results always implys the others.
We also mention the relationships between the access control function
and this dependency.

We also show an example of the access control function and the \testtool
dependency under a user transaction model, which is similar to actual 
CAD databases.

Finally we consider several extensions of the access control functions
and the \testtool dependencies.
We also consider a concept of user hierarchy and change notification
methods on our data management model.
\end{abstract}
